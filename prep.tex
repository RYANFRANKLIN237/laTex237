\documentclass[12pt,a4paper]{article} % Document class with font size and page size

% Packages for advanced features
\usepackage{graphicx}    % For including images
\usepackage{amsmath}     % For mathematical symbols
\usepackage{geometry}    % For page layout customization
\geometry{margin=1in}    % 1-inch margins
\usepackage{natbib}      % For bibliography management
\usepackage{hyperref}    % For hyperlinks in the document
\usepackage{fancyhdr}    % For custom headers and footers

% Title and Author details
\title{A Sample Report on LaTeX} 
\author{Your Name \\ \texttt{your.email@example.com}}
\date{\today} % Automatically generates today's date

% Begin Document
\begin{document}

% Title Page
\maketitle
\thispagestyle{empty} % Removes page number from title page

% Abstract
\begin{abstract}
This document demonstrates the creation of a structured report using \LaTeX. It includes various sections such as the introduction, figures, tables, equations, and bibliographic references. The document uses packages like \texttt{graphicx}, \texttt{amsmath}, \texttt{natbib}, and others.
\end{abstract}

% Table of Contents
\tableofcontents
\newpage % Page break after Table of Contents

% Section 1: Introduction
\section{Introduction}
\LaTeX~is a document preparation system used widely in academia. It is ideal for creating technical and scientific documents. This sample will illustrate:
\begin{itemize}
    \item How to structure a document.
    \item How to include figures and tables.
    \item How to use bibliographic references.
\end{itemize}

% Section 2: Figures
\section{Including Figures}
You can include figures using the \texttt{graphicx} package. Figure~\ref{fig:sample} is an example.

\begin{figure}[h]
    \centering
    \includegraphics[width=0.5\textwidth]{example-image-a} % Replace with actual image file name
    \caption{A Sample Figure}
    \label{fig:sample}
\end{figure}

% Section 3: Equations
\section{Mathematical Equations}
Mathematical equations are written using the \texttt{amsmath} package. For example:
\begin{equation}
    E = mc^2
    \label{eq:einstein}
\end{equation}
Equation~\ref{eq:einstein} represents the mass-energy equivalence.

% Section 4: Tables
\section{Tables}
Tables can be created as shown below.

\begin{table}[h]
    \centering
    \begin{tabular}{|c|c|c|}
    \hline
    Column 1 & Column 2 & Column 3 \\ \hline
    A        & B        & C        \\ \hline
    1        & 2        & 3        \\ \hline
    \end{tabular}
    \caption{A Sample Table}
    \label{tab:sample}
\end{table}

% Section 5: Bibliography
\section{Bibliography}
References are managed using the \texttt{natbib} package. Here is an example citation~\citep{shor1994}.

\bibliographystyle{plain}
\bibliography{references} % Create a "references.bib" file with bibliographic entries

% Sample Bibliographic Entry (in references.bib file)
% @book{lamport1994latex,
%   title={LaTeX: A document preparation system},
%   author={Lamport, Leslie},
%   year={1994},
%   publisher={Addison-Wesley}
% }

% Footer Setup
\newpage
\pagestyle{fancy}
\fancyhf{}
\fancyhead[L]{\textit{A Sample LaTeX Report}}
\fancyfoot[C]{\thepage}

% Conclusion
\section{Conclusion}
This document has demonstrated how to create a complete report using \LaTeX, including sections, tables, figures, equations, and references. Mastering these skills will help you prepare professional documents for academic or research purposes.

\end{document}
